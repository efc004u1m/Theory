\documentclass[14pt]{article}
\usepackage[utf8]{inputenc}
\usepackage[margin=0.6in]{geometry}
\usepackage[document]{ragged2e}
\usepackage{enumitem}
\usepackage{textcomp}
\usepackage{lmodern}
\usepackage{easylist}
\usepackage{listings}
\usepackage{xcolor}
\usepackage{setspace}
\setstretch{1.25}
\usepackage[most]{tcolorbox}
\usepackage{fancyhdr}

\pagestyle{fancy}
\renewcommand{\headrulewidth}{0pt}
\fancyhead[]{}
\fancyfoot[L]{Entry No: }

% set the default code style
\lstset{
    frame=tb, % draw a frame at the top and bottom of the code block
    tabsize=4, % tab space width
    showstringspaces=false, % don't mark spaces in strings
    numbers=left, % display line numbers on the left
    commentstyle=\color{green}, % comment color
    keywordstyle=\color{blue}, % keyword color
    stringstyle=\color{red} % string color
}
\newtcolorbox{myframe}[1][]{
  enhanced,
  arc=0pt,
  outer arc=0pt,
  colback=white,
  boxrule=0.8pt,
  #1
}

\begin{document}
\large
EFC004U1M: Data Structures and Algorithms (IIT Jammu, Semester-II-2019-20)

\begin{flushright}
Mid Term Examination\\
\vspace{1em}
Name: \underline{\hspace{6cm}} \\
\vspace{1em}
Entry Number: \underline{\hspace{6cm}} \\
\end{flushright}

\RaggedRight

There are 5 questions for a total of 16 points. 
\hfill Maximum Score: 15 points \\*
Answer in the spaces/box provided. \\* Write your \textbf{Entry No.} at bottom left on each page.
\hfill Maximum Time: 2 hours
\noindent\rule{\textwidth}{1pt}

\section{State True or False:}
\begin{enumerate}[label=\alph*)]
    \item (\textonequarter\ point) A function can return any type of value in C++ regardless of the type specified in \\* declaration.
    \hfill (a) \underline{\hspace{5cm}}
    
    \item (\textonequarter\ point) You are given a snippet of code declaring some variables:
    \begin{lstlisting}[language=C++, title={Part (b)}]
    int x = 20;
    int y = 10;
    int z = 4;
    \end{lstlisting}
    The expression \underline{$!(x + y > 8*z)$} evaluates to 
    \hfill (b) \underline{\hspace{5cm}}
    
    \item (\textonequarter\ point) A pretest loop is one in which the conditional statement is checked before \\* the execution of the block of code. While loop is NOT a pretest loop.\ \hfill (c) \underline{\hspace{3cm}}
    
    \item (\textonequarter\ point) If a function returns no value, the return type must be declared as void. \\*
    \hfill (d) \underline{\hspace{5cm}}
    
    \item (\textonequarter\ point) A pointer variable can't point to other pointer variables. \\*
    \hfill (e) \underline{\hspace{5cm}}
    
    \item (\textonequarter\ point) All objects of a class share all data members, private and public, of a class. \\*
    \hfill (f) \underline{\hspace{5cm}}
    
    \item (\textonequarter\ point) A pointer stores the address of the variable whereas a reference is just an alias for that variable.
    \hfill (g) \underline{\hspace{5cm}}
    
    \item (\textonequarter\ point) The tightest upper bound that represents the \textbf{no. of swaps} required to sort $n$ numbers using \textbf{selection sort} is $O(n^2)$. In selection sort, we first pick the min. element and swap with the first position, then the $2^{nd}$ min is swapped with $2^{nd}$ position and so on. \\*
    \hfill (h) \underline{\hspace{3cm}}
\end{enumerate}

\section{Fill in the blanks:}
\begin{enumerate}[label=\alph*)]
    \item (\textonequarter\ point) The type of 17.23 in C++ can be \underline{\hspace{10cm}}.
    
    \item (\textonequarter\ point) If originally $x = 1$ and $y = 0.5$, the value of $answer$ after executing the following snippet of code will be \underline{\hspace{5em}}.
    \begin{lstlisting}[language=C++, title={Part (b)}]
    int answer;
    x *= 5;
    answer = -x + (-y);
    \end{lstlisting}
    
    \item (\textonequarter\ point) $int\ arr[4][3]$ is an integer array with \underline{\hspace{5cm}} bytes of memory. (int takes 32 bits of size).
    
    \item (\textonequarter\ point) Find the asymptotic relationship ($\mathcal{O}$/$\Omega$/$\Theta$). $n^k$ is \underline{\hspace{1cm}}($c^n$). Assume that $k >= 1$ and $c > 1$ are constants.
    
    \item (\textonequarter\ point) A sorting algorithm takes 1 second to sort 1,000 items on your local machine. The time it will take to sort 10,000 items if you believe that the algorithm takes time proportional to $n^2$ is $\underline{\hspace{5cm}}$.
    
    \item (\textonequarter\ point)
    The following function computes the maximum value contained in an integer array $p[]$ of size $n\ (n \geq 1)$. Fill the missing loop condition.
    \begin{lstlisting}[language=C++, title={Part (f)}]
    int max(int *p, int n) {
        int a = 0, b = n - 1;
        while (______________) {
            if (p[a] <= p[b]) {a = a + 1;}
            else              {b = b - 1;}
        }
        return p[a];
    }
    \end{lstlisting}
    
\end{enumerate}

\section{Find the output: (answer in the box)}
\begin{enumerate}[label=\alph*)]
    \item (\textonequarter\ point)
    \begin{lstlisting}[language=C++, title={Part (a)}]
    int ar[4] = {2, 4, 6, 8};
    ar[0] = 27;
    ar[3] = ar[2];
    cout << ar[0] << " and " << ar[3] << endl;
    \end{lstlisting}
    \framebox(500,30){}
    
    \item (\textonequarter\ point)
    \begin{lstlisting}[language=C++, title={Part (b)}]
    class Parent {
    public:
        void PPrint() {
            cout << "Parent print ";
        }
    };
    class Child: public Parent {
    public:
        void CPrint() {
            cout << "Child print ";
        }
    };
    int main () {
        Child c;
        c.PPrint();
    }
    \end{lstlisting}
    \framebox(500,30){}
    
    \item (\textonequarter\ point)
    \begin{lstlisting}[language=C++, title={Part (c)}]
    #include <iostream>
    using namespace std;
    int main () {
        int a[4][5] = {{1, 2, 3, 4, 5},
                      {6, 7, 8, 9, 10},
                      {11, 12, 13, 14, 15},
                      {16, 17, 18, 19, 20}};
        cout << *(*(a+**a+2)+3) << endl;
    }
    \end{lstlisting}
    \framebox(500,20){}
    
    \item (\textonequarter\ point)
    Consider the following C++ functions: \\*
    \noindent\begin{minipage}{.45\textwidth}
    \begin{lstlisting}[language=C++, title={Part (d) - I}]
    int tob(int b, int* arr) {
        int i;
        for (i = 0; b > 0; ++i){
            if (b % 2) 
                arr[i] = 1;
            else
                arr[i] = 0;
            b = b/2;
        }
        return (i);
    }
    \end{lstlisting}
    \end{minipage}\hfill
    \begin{minipage}{.45\textwidth}
    \begin{lstlisting}[language=C++, title={Part (d) - II}]
    int pp(int a, int b) {
        int arr[20];
        int i, tot = 1, ex, len;
        ex = a;
        len = tob(b, arr);
        for (i = 0; i < len; ++i) {
            if (arr[i] == 1) {
                tot = tot * ex;
            }
            ex = ex * ex;
        }
        return (tot);
    }
    \end{lstlisting}
    \end{minipage}
    \\* The value returned by $pp(2, 10)$ is: \\*
    \framebox(500,20){}
\end{enumerate}

\section{Fix the bug in the following: (mention line number(s) and the fix)}
\begin{enumerate}[label=\alph*]
    \item (\textonequarter\ point)
    \begin{lstlisting}[language=C++, title={Part (a)}]
    class Quiz {
        int score;
    }
    int main () {
        Quiz mid_term;
        cout << mid_term.score << endl;
        return 0;
    }
    \end{lstlisting}
    \framebox(500,50){}
    
    \item (\textonequarter\ point)
    \begin{lstlisting}[language=C++, title={Part (b)}]
    void mystery(int *ptra, int *ptrb) {
        int *temp;
        temp = ptrb;
        ptrb = ptra;
        ptra = temp;
    }
    int main () {
        int a = 2020, b = 0, c = 4, d = 42;
        mystery(a, b);
        if (a < c) {
            mystery(c, a);
        }
        mystery(a, d);
        cout << a << endl;
    }
    \end{lstlisting}
    \framebox(500,100){}
    
\end{enumerate}

\section{Answer the following:}
\begin{enumerate}[label=\alph*)]
    \item ($2$\ points)
    Consider the following C++ code segment. Let $T(n)$ denote the no. of times the for loop is executed by the program on input $n$. Which of the following claims are true? (Also give explanation for each).
    \begin{lstlisting}[language=C++, title={Part (a)}]
    int isPrime(int n) {
        for (int i = 2; i <= sqrt(n); ++i) {
            if (n % i == 0) {
                return 0;
            }
        }
        return 1;
    }
    \end{lstlisting}
    \begin{enumerate}[label=\roman*]
        \item $T(n) = \mathcal{O}(\sqrt{n})$
        \item $T(n) = \Omega(\sqrt{n})$
        \item $T(n) = \Omega(1)$
        \item $T(n) = \mathcal{O}(n)$
    \end{enumerate}
    \framebox(500,300){}
    
    \item ($2$\ points)
    Write a function to perform integer division without using either the $/$ or $*$ operators. Your function should take in two arguments, the numerator and the denominator. It returns the integer quotient. Assume that both inputs to the function are positive. \\*
    
    \framebox(500,280){}
    \framebox(500,250){}
    \item ($2$\ points)
    Suppose the following algorithm is used to evaluate the polynomial \\*
    \[p(x) = a_{n}x^{n} + a_{n-1}x^{n-1} + \cdots + a_{1}x + a_{0} \]
    \begin{lstlisting}[language=C++, title={Part (c) - I}]
    int p = a[0];
    int xpower = 1;
    for (int i = 1; i <= n; ++i) {
        xpower = x*xpower;
        p = p + a[i]*xpower;
    }
    \end{lstlisting}
    What is the no. of multiplications done in the worst-case? Also, how many additions? \\*
    \framebox(500,50){}
    Now, consider the following algorithm:
    \begin{lstlisting}[language=C++, title={Part (c) - II}]
    int p = 0;
    for (int i = n; i >= 0; --i) {
        p = (p*x) + a[i];
    }
    \end{lstlisting}
    Does the above algorithm evaluate the polynomial correctly for a given value of $x$ and the coefficients $a[0 \cdots n]$? How many additions and multiplications does it require? What are the time complexities of both the algorithms? \\*
    \framebox(500,250){}
    
    \item ($3$\ points)
    You are given an array $A[0 \cdots (n-1)]$, where each element of the array represents a vote in the election. Assume each vote is given as an integer representing the ID of the chosen candidate. \textbf{Write a function that determines who wins the election (the ID of the winning candidate)}. Assume the inputs to the function are the integer array and n (no. of votes), and it returns the integer ID of the winning candidate. \\*
    Also, analyze the time complexity of your solution. \\*
    \framebox(500,260){}
    \framebox(500,200){}
    Can you improve the complexity of your solution? Assume that you know sorting an array of size $n$ takes $O(n\ logn)$ time. Explain your new algorithm. (No need to write code).
    
    \framebox(500,200){}
    
    \item ($1$\ point)
    Solve the following recurrence relation and write the exact value of $T(n).$ Show calculations. (DO NOT USE MASTER'S THEOREM).
    \[T(n) = 2 \times T(n/2) + n^2; \ T(1) = 1 \]
    (assume that $n$ is a power of $2$). Finally, express $T(n)$ in Big-O notation.
    
    \framebox(500,100){}
    \framebox(500,220){}
    
    \item ($1$\ point)
    You need to write a recurrence relation that leads to the time complexity of $\mathcal{O}(nloglogn)$. Also, write a recursive function corresponding to it. (Mention any assumptions you have made).
    \framebox(500,400){}
\end{enumerate}
\end{document}