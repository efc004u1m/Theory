\documentclass[14pt]{article}
\usepackage[utf8]{inputenc}
\usepackage[margin=0.6in]{geometry}
\usepackage[document]{ragged2e}
\usepackage{enumitem}
\usepackage{textcomp}
\usepackage{lmodern}
\usepackage{easylist}
\usepackage{listings}
\usepackage{xcolor}
\usepackage[most]{tcolorbox}

\newtcolorbox{myframe}[1][]{
  enhanced,
  arc=0pt,
  outer arc=0pt,
  colback=white,
  boxrule=0.8pt,
  #1
}

% set the default code style
\lstset{
    frame=tb, % draw a frame at the top and bottom of the code block
    tabsize=4, % tab space width
    showstringspaces=false, % don't mark spaces in strings
    numbers=left, % display line numbers on the left
    commentstyle=\color{green}, % comment color
    keywordstyle=\color{blue}, % keyword color
    stringstyle=\color{red} % string color
}

\begin{document}
\large
EFC004U1M: Data Structures and Algorithms (IIT Jammu, Semester-II-2019-20)

\begin{flushright}
Theory Assignment 1 \\
\vspace{1em}
Name: \underline{\hspace{6cm}} \\
\vspace{1em}
Entry Number: \underline{\hspace{6cm}} \\
\end{flushright}

\RaggedRight
There are 3 questions for a total of 5 points. \\*
Answer in the spaces/box provided. \\*
\noindent\rule{\textwidth}{1pt}

\section{Answer the following:}
\begin{enumerate}[label=\alph*)]
    \item (\textonequarter\ point) \underline{State true or false:} Let $f(n) = 5n2^{n} + 3^{n}$ and $g(n) = n3^{n}.$ Then $f(n) = \mathcal{O}(g(n)).$ \\ \vspace{1em}
    \begin{flushright}
    (a) \underline{\hspace{2cm}True\hspace{2cm}}
    \end{flushright}
    
    \item (\textonequarter\ point) \underline{State true or false:} Let $f(n) = 5n2^{n} + 3^{n}$ and $g(n) = n3^{n}.$ Then $g(n) = \mathcal{O}(f(n)).$ \\ \vspace{1em}
    \begin{flushright}
    (b) \underline{\hspace{2cm}False\hspace{2cm}}
    \end{flushright}
    
    \item (\textonequarter\ point) Express the running time of the algorithm below in big-\mathcal{O} notation. \\
    \fbox{\begin{minipage}{30em}
    Algorithm1(A:array, N:int)
    \begin{itemize}
        \item[$-$] for $i$ = $1$ to $N$
            \begin{itemize}
                \item [$-$] for $j$ = $2*i$ to $N$
                    \begin{itemize}
                        \item [$-$] A[i] $\leftarrow$ $A[j] + 1$
                    \end{itemize}
            \end{itemize}
    \end{itemize}
    \end{minipage}}
    \begin{flushright}
    (c) \underline{\hspace{2cm}$\mathcal{O}(N^{2})$\hspace{2cm}}
    \end{flushright}
    
    \item (\textonequarter\ point) \underline{Fill in the blanks:} Let $f(n) = 2n!+5*3^{n} + 4^{n} + n^7$ . Then $f(n)$ in $\Theta$ notation is \\ \vspace{1em}
    \begin{flushright}
    (d) \underline{\hspace{2cm}$\Theta(N!)$\hspace{2cm}}
    \end{flushright}
\end{enumerate}


\section{Find the time complexity of following piece of codes in Big-\mathcal{O} notation (also give explanation):}
\begin{enumerate}[label=\alph*)]
    \item ($1$ point)
    
    \begin{lstlisting}[language=C++, title={Part (a)}]
    void function(int n) {
        int i = 1;
        int s = 1;
        int count = 0;
        while (s <= n) {
            i++;
            s += i;
            ++count;
        }
    }
    \end{lstlisting}
    
    \begin{myframe}[width=500pt,height=250pt,top=10pt,bottom=10pt,left=10pt,right=10pt,arc=10pt,auto outer arc]
    The value of $i$ increases by $1$ for each iteration. \\*
    The value of $s$ after $j^{th}$ iteration of the loop, is the sum of first $j+1$ positive integers. \\*
    If $k-1$ is the total iterations taken by the program, the while loop terminates if:
    \[1 + 2 + \cdots + k = \frac{k(k + 1)}{2} > n\]
    \[\implies k^2 + k - 2n > 0 \]
    \[\implies k > \frac{-1 + \sqrt{8n}}{2} \] 
    \begin{flushright}
    (ignore other root, since $k > 0$)
    \end{flushright}
    Thus, the complexity of this program depends upon the no. of iterations of the loop, which is $\mathcal{O}(k)$, or 
    \[ \boxed{\mathcal{O}(\sqrt{N})} \]
    
    \end{myframe}
    \item ($1$ point)
    \begin{lstlisting}[language=C++, title={Part (b)}]
    void function(int n) {
        int i, j, count = 0;
        for (i = n/2; i <= n; ++i) {
            for (j = 1; j <= n; j *= 2) {
                ++count;
            }
        }
    }
    \end{lstlisting}
    \begin{myframe}[width=500pt,height=280pt,top=10pt,bottom=10pt,left=10pt,right=10pt,arc=10pt,auto outer arc]
    The outer loop runs $n - \frac{n}{2} + 1 = \frac{n}{2} + 1$ times, i.e. $\mathcal{O}(n)$ \\*
    Let us assume that the inner loop runs for $k$ iterations.
    At every iteration, the value of $j$ is doubled. So, at $k^{th}$ iteration, $j$ becomes: $2^{k}$. For the loop to terminate,
    \[ 2^{k} > n\]
    \[ \implies k > log\ n\]
    Let $k = log\ n + c$ where $c$ is some constant, we can easily show that $k$ is $\mathcal{O}(log\ n)$.
    Hence, the time complexity of the program is:
    \[ \boxed{\mathcal{O}(n\ log\ n)} \]
    \end{myframe}
    
    \item ($1$ point)
    \begin{lstlisting}[language=C++, title={Part (c)}]
    void function(int n) {
        if (n == 1) {
            return;
        }
        for (int i = 1; i <= n; ++i) {
            for (int j = 1; j <= n; ++j) {
                cout << "*";
                break;
            }
        }
    }
    \end{lstlisting}
    \begin{myframe}[width=500pt,height=120pt,top=10pt,bottom=10pt,left=10pt,right=10pt,arc=10pt,auto outer arc]
    The outer loop runs $\mathcal{O}(n)$ times, whereas inner loop runs $\mathcal{O}(1)$ time. \\*
    Thus, the complexity of the program is:
    \[ \boxed{\mathcal{O}(n)} \]
    \end{myframe}
\end{enumerate}

\section{Answer the following question with explanation:}
($1$ point) Which of the following 3 claims are correct:

\begin{enumerate}
    \item $(n + k)^{m} = \Theta(n^m)$ where $k$ and $m$ are constants.
    \item $2^{n + 1} = \mathcal{O}(2^{n})$
    \item $2^{2n + 1} = \mathcal{O}(2^{n})$
\end{enumerate}

\begin{myframe}[width=500pt,height=250pt,top=10pt,bottom=10pt,left=10pt,right=10pt,arc=10pt,auto outer arc]
Claims $1$ and $2$ are correct. \\*
\begin{enumerate}
    \item To show that $(n + k)^{m}$ is $\Theta(n^{m})$ we must show that it is both $\mathcal{O}(n^m)$ and $\Omega(n^m)$.
    \[(n + k)^{m} \leq c_{1} (n^{m}) \] 
    We can choose $c_{1} = k^{m}$. Thus, $(n + k)^{m}$ is $\mathcal{O}(n^{m})$.
    \[c_{2} (n^{m}) \leq (n + k)^{m} \] 
    We can choose $c_{2} = 1$. Thus, $(n + k)^{m}$ is $\Omega(n^{m})$.
    \item 
    \[ 2^{n+1} = 2 \times (2^{n}) = \mathcal{O}(2^{n})\]
    \item 
    \[ 2^{2n+1} = 2 \times (2^{2n}) = 2 \times (4^{n})\]
    It is \textbf{not} $\mathcal{O}(2^{n})$ as there does not exist $c$ and $n_{0}$ for which $2\times4^{n} \leq c\times2^{n}$, $\forall\ n \geq n_{0}$.
    
\end{enumerate}

\end{myframe}
\end{document}