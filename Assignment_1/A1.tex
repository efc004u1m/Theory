\documentclass[14pt]{article}
\usepackage[utf8]{inputenc}
\usepackage[margin=0.6in]{geometry}
\usepackage[document]{ragged2e}
\usepackage{enumitem}
\usepackage{textcomp}
\usepackage{lmodern}
\usepackage{easylist}
\usepackage{listings}
\usepackage{xcolor}
% set the default code style
\lstset{
    frame=tb, % draw a frame at the top and bottom of the code block
    tabsize=4, % tab space width
    showstringspaces=false, % don't mark spaces in strings
    numbers=left, % display line numbers on the left
    commentstyle=\color{green}, % comment color
    keywordstyle=\color{blue}, % keyword color
    stringstyle=\color{red} % string color
}

\begin{document}
\large
EFC004U1M: Data Structures and Algorithms (IIT Jammu, Semester-II-2019-20)

\begin{flushright}
Theory Assignment 1 \\
\vspace{1em}
Name: \underline{\hspace{6cm}} \\
\vspace{1em}
Entry Number: \underline{\hspace{6cm}} \\
\end{flushright}

\RaggedRight
There are 3 questions for a total of 5 points. \\*
Answer in the spaces/box provided. \\*
\noindent\rule{\textwidth}{1pt}

\section{Answer the following:}
\begin{enumerate}[label=\alph*)]
    \item (\textonequarter\ point) \underline{State true or false:} Let $f(n) = 5n2^{n} + 3^{n}$ and $g(n) = n3^{n}.$ Then $f(n) = O(g(n)).$ \\ \vspace{1em}
    \begin{flushright}
    (a) \underline{\hspace{5cm}}
    \end{flushright}
    
    \item (\textonequarter\ point) \underline{State true or false:} Let $f(n) = 5n2^{n} + 3^{n}$ and $g(n) = n3^{n}.$ Then $g(n) = O(f(n)).$ \\ \vspace{1em}
    \begin{flushright}
    (b) \underline{\hspace{5cm}}
    \end{flushright}
    
    \item (\textonequarter\ point) Express the running time of the algorithm below in big-O notation. \\
    \fbox{\begin{minipage}{30em}
    Algorithm1(A:array, N:int)
    \begin{itemize}
        \item[$-$] for $i$ = $1$ to $N$
            \begin{itemize}
                \item [$-$] for $j$ = $2*i$ to $N$
                    \begin{itemize}
                        \item [$-$] A[i] $\leftarrow$ $A[j] + 1$
                    \end{itemize}
            \end{itemize}
    \end{itemize}
    \end{minipage}}
    \begin{flushright}
    (c) \underline{\hspace{6cm}}
    \end{flushright}
    
    \item (\textonequarter\ point) \underline{Fill in the blanks:} Let $f(n) = 2n!+5*3^{n} + 4^{n} + n^7$ . Then $f(n)$ in $\theta$ notation is \\ \vspace{1em}
    \begin{flushright}
    (d) \underline{\hspace{5cm}}
    \end{flushright}
\end{enumerate}


\section{Find the time complexity of following piece of codes in Big-O notation (also give explanation):}
\begin{enumerate}[label=\alph*)]
    \item ($1$ point)
    
    \begin{lstlisting}[language=C++, title={Part (a)}]
    void function(int n) {
        int i = 1;
        int s = 1;
        int count = 0;
        while (s <= n) {
            i++;
            s += i;
            ++count;
        }
    }
    \end{lstlisting}
    \framebox(500,250){}
    
    \item ($1$ point)
    \begin{lstlisting}[language=C++, title={Part (b)}]
    void function(int n) {
        int i, j, count = 0;
        for (i = n/2; i <= n; ++i) {
            for (j = 1; j <= n; j *= 2) {
                ++count;
            }
        }
    }
    \end{lstlisting}
    \framebox(500,280){}
    
    \item ($1$ point)
    \begin{lstlisting}[language=C++, title={Part (c)}]
    void function(int n) {
        if (n == 1) {
            return;
        }
        for (int i = 1; i <= n; ++i) {
            for (int j = 1; j <= n; ++j) {
                cout << "*";
                break;
            }
        }
    }
    \end{lstlisting}
    \framebox(500,120){}
\end{enumerate}

\section{Answer the following question with explanation:}
($1$ point) Which of the following 3 claims are correct:

\begin{enumerate}
    \item $(n + k)^{m} = \theta(n^m)$ where $k$ and $m$ are constants.
    \item $2^{n + 1} = O(2^{n})$
    \item $2^{2n + 1} = O(2^{n})$
\end{enumerate}

\framebox(500,250){}
\end{document}