\documentclass[14pt]{article}
\usepackage[utf8]{inputenc}
\usepackage[margin=0.6in]{geometry}
\usepackage[document]{ragged2e}
\usepackage{enumitem}
\usepackage{textcomp}
\usepackage{lmodern}
\usepackage{easylist}
\usepackage{listings}
\usepackage{xcolor}
\usepackage{url}
\usepackage[most]{tcolorbox}
\usepackage{hyperref}

% set the default code style
\lstset{
    frame=tb, % draw a frame at the top and bottom of the code block
    tabsize=4, % tab space width
    showstringspaces=false, % don't mark spaces in strings
    numbers=left, % display line numbers on the left
    commentstyle=\color{green}, % comment color
    keywordstyle=\color{blue}, % keyword color
    stringstyle=\color{red} % string color
}
\newtcolorbox{myframe}[1][]{
  enhanced,
  arc=0pt,
  outer arc=0pt,
  colback=white,
  boxrule=0.8pt,
  #1
}

\begin{document}
\large
EFC004U1M: Data Structures and Algorithms (IIT Jammu, Semester-II-2019-20)

\begin{myframe}[width=500pt,height=40pt,top=10pt,bottom=10pt,left=10pt,right=10pt,arc=10pt,auto outer arc]
\Large \center{Assignment 2: Stack and its Applications}
\end{myframe}

\section{[2 Points] Implementing Stack}
In this part, you will implement a \textbf{growable stack} using arrays in C++. \\
You start with an array of size = 1. Every time the stack gets full, you copy the data to an array of double the previous size, and the new array now represents the stack. \\*
To do so, you will have to use a dynamic array (using pointers) and allocate memory using the \textbf{new} operator.

\begin{itemize}
    \item Write a C++ class \textbf{MyStack} in file \textbf{MyStack.cpp} that implements the above stack with following public methods.
    \begin{itemize}
        \item \textbf{void push(int item)} : Pushes the \textbf{item} onto the top of the stack.
        \item \textbf{int pop()} : Removes the element at the top of the stack and returns it.
        \item \textbf{int peek()} : Returns the top element without removing from top of the stack.
        \item \textbf{bool empty()} : Returns $true$ is the stack is empty and $false$ otherwise.
    \end{itemize}
    All other methods or variables in your implementation should be \textit{private}.
\end{itemize}

\section{[3 Points] Evaluate Arithmetic Expressions}

Infix and postfix notations are two different ways of writing arithmetic expressions. \\*
Please see the URL below for more details on the two notations. \\*

\url{http://www.cs.man.ac.uk/~pjj/cs2121/fix.html}


Write a program to evaluate infix expressions \textbf{using the stack created in problem 1}. Your program should read the expression as input string and evaluate it in a single pass. \\*

(\textbf{Hint}: You have to use the infix to postfix conversion algorithm and postfix evaluation algorithm using two stacks - one stack for the operands and one stack for the operators. The infix to postfix conversion algorithm is explained in the following URL: \\*
\url{http://csis.pace.edu/~wolf/CS122/infix-postfix.htm} \\*
The postfix evaluation algorithm is explained in the following URL: \\*
\url{https://scriptasylum.com/tutorials/infix_postfix/algorithms/postfix-evaluation/}

The infix expression can contain:
\begin{enumerate}
    \item Integers
    \item `+' and `*' operators
    \item `(' and `)' parentheses
\end{enumerate}

As usual, * operator has higher precedence than + operator. \\*
Both * and + are left associative.

\begin{itemize}
    \item Name your file as \textbf{InfixEvaluate.cpp}.
    \item The input to a program is a string, for e.g ``2 + 3 * (5 + 6)''
    \item The output to a program is an integer, for e.g ``35''.
    \item Do not print anything else to the screen.
\end{itemize}
\end{document}
